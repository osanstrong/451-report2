\documentclass[12pt]{article}
\usepackage{graphicx} % Required for inserting images

\title{Lab 2: Introduction to Lab Equipment and Counting Statistics
}
\author{Owen Strong \\
\itshape Prof.: Angela DiFulvio\\
\itshape TA: Kholod Mahmoud \\
\itshape TA: Schaffer Bauer \\
\itshape TA: Justin Jia \\
}
\date{7 February 2026}

\usepackage[
    style=ieee
]{biblatex}
\bibliography{refs.bib}

\usepackage[letterpaper, margin=1in]{geometry}
\newcommand{\figwidth}{0.75\linewidth}

\usepackage{placeins} % FloatBarrier
\usepackage{amsmath} % align
\begin{document}

\maketitle
\begin{abstract}

\end{abstract}

\section{Introduction and Background}\label{sec:intr}

\subsection{Geiger-Muller (GM) Detector}

\subsection{Multichannel Analyzer (MCA)}

\subsection{Counting Statistics}
A key variable of interest in this laboratory is the number of counts measured by the system in a fixed period of time.
This can, theoretically, be modeled by a Poisson Distribution, which describes the number of random events taking place in a fixed time period.\\

A Poisson distribution $Pois(f)$ for a time period is parameterized by $f$, the expected frequency per that time period.
If we assume the likelihood of observing some number of counts can be modeled by this distribution, then we expect those measurements to vary by this distribution's standard deviation $\sigma_P$\cite{kotz_encyclopedia_2006}:
\begin{equation}
    \sigma_P = \sqrt{f}
\end{equation}
By assuming a theoretical frequency similar enough to that experimentally observed, uncertainty of a radiation count $N$ can thus be estimated as $\sigma_N=\sqrt{N}$.

\section{Experimental Procedure}\label{sec:methods}
Certain new equipment was used in this laboratory, to be detailed as follows:
\subsection{Geiger-Muller Detector}
In this laboratory, we use a Geiger-Muller Detector, henceforth referred to as the Detector. 
We were unable to locate identifying information on this device.

\subsection{Multichannel Analyzer}
We use an Ortec EasyMCA MCA, with inventory number P10F79090. This MCA will henceforth be referred to the MCA, or Multichannel Analyzer.

\subsection{Radiation Source}
In this laboratory, we use a [DETAILS] source numbered 2037, and will henceforth be referred to as the Source.

\section{Experimental Results}\label{sec:results}

%%%% Experiment 6: Multichannel Analyzer
\FloatBarrier\subsection{Experiment 6: Multichannel Analyzer}

% Plot of the acquired spectrum
\begin{figure}[!h]
    \centering
    \includegraphics[width=\figwidth]{figs/Exp6_4vSpectrum.png}
    \caption{The spectrum acquired from the Multichannel Analyzer with a pulse amplitude of 4V.}
    \label{fig:exp6-4v-spec}
\end{figure}

% "Plots of sample spectra as a function of the pulse amplitude"
\begin{figure}[!h]
    \centering
    \includegraphics[width=\figwidth]{figs/Exp6_V_v_chan.png}
    \caption{The peak centroid of spectra acquired from the Multichannel Analyzer with pulse amplitudes of varying voltage.}
    \label{fig:exp6-v-v-chan}
\end{figure}

%%%% Experiment 6: Repeated Measurements
\FloatBarrier\subsection{Statistical Uncertainty of a Series of Independent Measurements}

Two series of count measurements were performed, of 20 trials lasting 30 seconds each and of 200 trials lasting 5 seconds each.
For both data series, we calculate the sample mean and standard deviation, and use Method of Moments (MoM) to fit experimental results to theoretical Normal, Poisson, and Binomial distributions.\\

For each series, we calculate experimental mean $\bar{x}$ (where $n\in\{20,200\}$ is the number of measurements, and $x_i$ the (count) value measured for trial $i$):
\begin{equation}
    \bar{x} = \frac{1}{n}\sum_{i=1}^{n}x_i
\end{equation}
We then calculate the deviation $d_i$ of each point $x_i$:
\begin{equation}
    d_i = x_i - \bar{x}
\end{equation}
Using these deviations we calculate the sample variance\footnote{As this is the variance of a \emph{sample} and not a known population, the sample mean inherently depends on the specific sample itself, removing a degree of freedom. We thus normalize deviation by $n-1$ rather than by $n$.} $s^2$:
\begin{equation}
   s^2 = \frac{1}{n-1}\sum_{i=1}^{n}d_i 
\end{equation}
And sample standard deviation:
\begin{equation}
    s = \sqrt{s^2}
\end{equation}
Using these two `moments', we fit theoretical Normal, Poisson, and Binomial distributions to the data using Method of Moments\cite{kotz_encyclopedia_2006}.
For the Poisson distribution $Pois(f)$, the theoretical mean $\mu$ is equivalent to the frequency parameter $f$, and we fit ($\hat{f}$) using the sample mean as such:
\begin{align}
    \mu &= f = \bar{x}\\
    \hat{f} &= \bar{x}
\end{align}
For the Normal distribution $Norm(\mu, \sigma)$ the moments mean $\mu$ and standard deviation $\sigma$ are themselves parameters, and may be fit to sample values as such:
\begin{align}
    \hat{\mu} &= \bar{x} \\
    \hat{\sigma} &= s \\
\end{align}
For the Binomial distribution $Binom(n, p)$ these moments are functions of parameters $n$ and $p$, and may be inverted to fit to sample moments:
\begin{align}
    \mu &= np = \bar{x} \\
    \sigma &= np(1-p) &= s \\
    &... \\\
    \hat{n} &= \mathrm{round}\left(\frac{\bar{x}^2}{\bar{x} - s^2}\right) \\
    \hat{p} &= \bar{x}/\hat{n}
\end{align}

Histograms of both series' (20 trials lasting 30 seconds each and 200 trials lasting 5 seconds each) results are displayed and compared to analogous histograms of theoretical distributions in Figures~\ref{fig:exp7-30s-hist-v-fits} and \ref{fig:exp7-5s-hist-v-fits}, respectively.

% Histogram of the counts as explained in the text.
% 20 trials, 30 seconds each
\begin{figure}[!h]
    \centering
    \includegraphics[width=\figwidth]{figs/Exp7_20x30_hist_v_fits.png}
    \caption{
        A histogram of the counts observed over 20 trials each lasting 30 seconds, compared to expected counts under theoretical Normal, Poisson, and Binomial distribution distributions.
        Theoretical distributions were fit using Method of Moments.
    }
    \label{fig:exp7-30s-hist-v-fits}
\end{figure}
% 200 trials, 5 seconds each
\begin{figure}[!h]
    \centering
    \includegraphics[width=\figwidth]{figs/Exp7_200x5_hist_v_fits.png}
    \caption{
        A histogram of the counts observed over 200 trials each lasting 5 seconds, compared to expected counts under theoretical Normal, Poisson, and Binomial distribution distributions.
        Theoretical distributions were fit using Method of Moments.
    }
    \label{fig:exp7-5s-hist-v-fits}
\end{figure}

Notably, the distribution of 200 5-second trials in Figure~\ref{fig:exp7-5s-hist-v-fits} much more closely resembles theoretical distributions, forming a stable bell curve in contrast to the bimodal distribution seen in Figure~\ref{fig:exp7-30s-hist-v-fits} of the 20 30-second trials.
Additionally, the Normal and Bimodal distributions almost entirely overlap; this is expected, as at high $n$, Binomial distributions approach that of a Normal distribution with equivalent mean and standard deviation\cite{kotz_encyclopedia_2006}.


% Experiment 8: The Accuracy of a Single Measurement
\FloatBarrier\subsection{Experiment 8: The Accuracy of a Single Measurement}

Detector pulses were counted over a period of 5 minutes while attached to the Source for 5 minutes, and again without the Source for 10 minutes.\\

Over the 5-minute (300 second) Source period, a gross count $n_G$ of 3469 was observed, with an uncertainty of $\sigma_G=\sqrt{n_G}=58.90$.
This yields a gross count rate $f_G=3469\times(300\mathrm{s})^{-1}=11.562\mathrm{s^{-1}}$.
Since the error of a scaled function $a\times A$ is equivalent to its error scaled $a\times \sigma_A$\cite{tsoulfanidis_measurement_2015}, this yields a frequency uncertainty $\sigma_{f,G}=300^{-1}\times58.90=0.196\mathrm{s^{-1}}$.\\

Over the 10-minute (600 second) Source-free period, a background count $n_B$ of 386 was observed, with an uncertainty of $\sigma_B=\sqrt{n_B}=19.647$. 
This yields a background count rate $f_B=386\times(600\mathrm{s})^{-1}=0.643\mathrm{s^{-1}}$, and an error $\sigma_{f,B}=600^{-1}\times19.647=0.00739\mathrm{s^{-1}}$.\\

Assuming infinite timing precision, we may estimate the net activity rate $f_N$ by subtracting $f_B$ from $f_G$:
\begin{align}
    f_N &= f_G - f_B \\
    &= 11.562 - 0.643 \\
    f_N &= 10.920 \mathrm{s^{-1}}
\end{align}
As the variance of a difference $A-B$ is equivalent to the sum of each quantity's variance $\sigma_A^2+\sigma_B^2$\cite{tsoulfanidis_measurement_2015}, we may quantify the uncertainty of this measurement:
\begin{align}
    \sigma_f &= \sqrt{\sigma_{f,G}^2 + \sigma_{f,B}^2} \\
    &= \sqrt{0.196^2 + 0.007^2} \\
    \sigma_f &= 0.1964 \mathrm{s^{-1}}\\
\end{align}

The Source's calibrated activity was 1.0 milliCurie\cite{npre_lab_2_manual_2025}, equivalent to $3.7\times10^{10}\mathrm{s^{-1}}\times\mathrm{\frac{1Ci}{10^6\mu Ci}}=37,000\mathrm{s^{-1}}$.
Notably, this is dramatically higher than the net \emph{observed} activity of $10.920\pm0.196 \mathrm{s^{-1}}$.


\FloatBarrier\section{Discussion}\label{sec:disc}
\section{Conclusions}\label{sec:conc}

% \bibliographystyle{sty/ieeetran}
\printbibliography
\end{document}

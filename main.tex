\documentclass[12pt]{article}
\usepackage{graphicx} % Required for inserting images

\title{Lab 2: Introduction to Lab Equipment and Counting Statistics
}
\author{Owen Strong \\
\itshape Prof.: Angela DiFulvio\\
\itshape TA: Kholod Mahmoud \\
\itshape TA: Schaffer Bauer \\
\itshape TA: Justin Jia \\
}
\date{7 February 2026}

\usepackage[
    style=ieee
]{biblatex}
\bibliography{refs.bib}

\usepackage[letterpaper, margin=1in]{geometry}
\newcommand{\figwidth}{0.75\linewidth}

\usepackage{placeins} % FloatBarrier
\usepackage{amsmath} % align
\usepackage{hyperref} % URLS
\usepackage{enumitem} % fine control over table listings

%%%% Shorthands
\newcommand{\micCi}{$\mathrm{\mu Ci}$} % Microcuries

\begin{document}
\pagenumbering{gobble}
\maketitle
\pagebreak
\tableofcontents
\pagebreak
\pagenumbering{arabic}
\begin{abstract}

\end{abstract}

\section{Introduction and Background}\label{sec:intr}
% In the detection of radiation, it is particularly crucial to understand the detector itself.
% There are several types of radiation detectors, each of which best suited for different environments and types of radiation.
% Of particular interest is the Geiger-Muller detector, a type of detector using low-pressure gas in a voltage gap to detect radiation\cite{tsoulfanidis_measurement_2015}.
% When a particle 
A key requirement in detecting radiation is the ability to interpret measurements.
In a perfect world, a person wishing to work with radiation would be perfectly and acutely aware of the nature of every material, understanding the presence and interaction of every particle of any kind of radiation.
In the real world, just as macroscopic phenomena and objects must be inferred from visual and sensory information, the precise behavior microscopic phenomena like radiation interaction must be inferred through devices which respond to those interactions.
The field of statistics exists to illustrate this gap and quantify exactly what can be estimated from limited physical measurements.\\

Consider the radioactivity of a material sample.
Even when modeling the sample as emitting a consistent, deterministic rate of particles, it is at the very least \emph{unreasonable} to know exactly when each particle is released, and that exact rate.
Instead, a radiation detector would produce a predictable stimulus response to some of this radiation.
A statistical model would then, based on a theory of how the overall ``population'' of radiation events behave and relate to a given ``sample'' of detections, predict something about that population based on a sample, and quantify exactly how certain one can be about that estimate. (Or at least, the related ``population'' of how the detector will respond to radiation).\\

This laboratory investigates the behavior of a Geiger Muller detection in response to a beta radiation source, and compares certain statistical models of its behavior in order to describe the behavior's uncertainty. \\

The rest of this report proceeds as follows:
In the remainder of the introduction, an overview of equipment and theory will be given.
In Section~\ref{sec:methods}, details on 

\subsection{Geiger-Muller (GM) Detector}
In this laboratory we use a Geiger-Muller (GM) Detector.
A GM detector (referred to by the textbook as a GM counter) is a gas-filled detector, usually cylindrical in shape, in which one radiation detection triggers an avalanche of charge interactions to produce a signal independent of the first gas atom ionized.\cite{tsoulfanidis_measurement_2015} \\

This laboratory only makes introductory use of a GM detector, and more background about their function will be provided in the next laboratory, which will explore aspects such as dead time in more detail.

\subsection{Multichannel Analyzer (MCA)}
A Multichannel Analyzer takes incoming pulses and categorizes them into several channels (sometimes referred to as bins) based on their amplitude.\cite{tsoulfanidis_measurement_2015}
These bins may be presented in the form of a histogram called a spectrum, where the x-axis corresponds to the channel number, and the y-axis the number of counts sorted into that channel.
By calibrating the multichannel analyzer, each of these bins may be associated with a specific voltage or quantity of interest, but in this laboratory we will only focus on introductory use of this device.

\subsection{Counting Statistics}
A key variable of interest in this laboratory is the number of counts measured by the system in a fixed period of time.
This can, theoretically, be modeled by a Poisson Distribution, which describes the number of random events taking place in a fixed time period.\\
\subsubsection{Poisson Distribution}
A Poisson distribution $Pois(\lambda)$ for a time period is parameterized by $\lambda$, the expected number of events during that time period.
The likelihood of a certain number $x$ of such events happening during one time period can be found by using the distribution's probability mass function $p(x;\lambda)$\cite{nist_1366_nodate}:
\begin{equation}
    p(x;\lambda) = \frac{e^{-\lambda}\lambda^x}{x!}, x\in\{0, 1, 2, ...\}
\end{equation}
If we assume the likelihood of observing some number of counts can be modeled by this distribution, then we expect those measurements to vary by this distribution's standard deviation $\sigma_P$\cite{nist_1366_nodate}:
\begin{equation}
    \sigma_P = \sqrt{\lambda}
\end{equation}
By assuming a theoretical frequency similar enough to that experimentally observed, uncertainty of a radiation count $N$ can thus be estimated as $\sigma_N=\sqrt{N}$.

\subsubsection{Normal Distribution}
Another distribution that may be of interest is the Normal Distribution.
The normal distribution can be used to model the results of a wide variety of random circumstances, and is parameterized directly by a distribution mean $\mu$ and standard deviation $\sigma$.
The differential probability density $f(x)$ of a given quantity $x$ being sampled from a normal distribution $Norm(\mu, \sigma)$ may be found \cite{nist_1366_nodate}:
\begin{equation}
    f(x) = \frac{e^{-(x-\mu)^2}/(2\sigma^2)}{\sigma\sqrt{2\pi}}
\end{equation}

\subsubsection{Binomial Distribution}
Finally, we consider a third potential distribution of interest, the Binomial Distribution. \\

Suppose rather than being modeled by independent random detections, the counts $x$ of a detector were better modeled as the number of \emph{successful} detections, where a \emph{fixed} number $n$ of events occur, but only some average proportion $p$ are successfully detected. 
The counts could then be modeled using a binomial distribution $Binom(n, p)$, with the probability mass function $p(x;n,p)$\cite{nist_1366_nodate}:
\begin{equation}
    \left(\frac{n!}{x!(n-x)!}\right)(p)^x(1-p)^{(n-x)}, x\in{0,1,2,...,n}
\end{equation}

This binomial distribution would have an expected mean of $np$, and standard deviation of $\sqrt{np(1-p)}$.\\

\subsection{Error Propagation}
Of course, it is not just the measurement itself which is of interest.
Given a 

\section{Experimental Procedure}\label{sec:methods}

In this section, we detail the equipment used and the experiments performed in this laboratory.
Further equipment details are given in Appendix~\ref{app:equip}.

\subsection{Equipment}
In this laboratory, we employed the following equipment:

\subsubsection{Module Rack}

We used a Canberra Model 2000 Module Rack. This component will be henceforth referred to as the module rack.

\subsubsection{Geiger-Muller Detector}
In this laboratory, we used a GM Detector, henceforth referred to as the detector. 
We were unable to locate identifying information on this device.

\subsubsection{Multichannel Analyzer}
We used an Ortec EasyMCA multichannel analyzer, which we will henceforth refer to as the MCA, to validate signal details.

\subsubsection{Radiation Source}
In this laboratory, we used a 2017 1.0 \micCi cesium-137 beta radiation source, and will henceforth refer to it as the beta source.

\subsubsection{Programmable Power Supply}
To supply the inverter which follows the Detector, we used a Caen N1470AL 2CH High Voltage Programmable Power Supply, henceforth referred to as the power supply.

\subsubsection{Pulser}
In this laboratory, we used an Ortec Model 480 Pulser, henceforth called the pulser.
A serial number tag was not attached externally, and the module could not be removed to identify the Illinois property number on its side.


\subsubsection{Oscilloscope}

In this laboratory, we used an Infiniivision DSO-X 2002A Oscilloscope.
The particular oscilloscope used in this experiment will henceforth be referred to as the oscilloscope.

\subsubsection{Counter}

We used an Ortec Model 871 Timer-Counter, which will henceforth be referred to as the counter.

\subsubsection{Preamplifier}
In this laboratory, we used an Ortec 142PC Preamplifier, henceforth referred to as the preamplifier.
No Illinois property number was identifiable on the Preamplifier.

\subsubsection{Amplifier}

We used an Ortec Model 590A Amplifier and Timing Single Channel Analyzer, henceforth referred to as the Amplifier.

\subsubsection{Computer}

In this laboratory, we used a Dell Precision 3630, henceforth referred to as the Computer.

\subsubsection{Software: Maestro}

To interpret and display output from the MCA, we used the Maestro software provided by Ortec.\cite{ortec_maestro_2012}

\subsection{Experiment 6: The Multichannel Analyzer}

In this experiment, we used the Pulser to replicate the output of a spectroscopy detector.
We fed its output to the MCA as shown in Figure~\ref{fig:mat-exp6-setup}.\\
\begin{figure}[!h]
    \centering
    \includegraphics[width=\figwidth]{figs/Mat0_Exp6-setup.png}
    \caption{The setup for Experiment 6. Sourced from lab manual 2.\cite{npre_lab_2_manual_2025}}
    \label{fig:mat-exp6-setup}
\end{figure}

We first logged into the lab computer and started Maestro, ensuring the MCA was connected to the computer before configuring the equipment as shown in Figure~\ref{fig:mat-exp6-setup}.
We set the Amplifier gain to its lowest value, and adjusted the Pulser gain to find the maximum attainable pulse voltage, then to achieve approximately 25\% of that pulse voltage.\\

We set the collect time in Maestro to 1 minute before pressing the Start Acquisition button, as shown in Figure~\ref{fig:start-acquisition}
We then clicked on the spectrum as shown in Figure~\ref{fig:mat-sample-spectrum} to show channel number and count, or right clicked to identify precise peak details as in Figure~\ref{mat-peak-info} to record details such as the centroid of a peak.\\

\begin{figure}
    \centering
    \includegraphics[width=\figwidth]{figs/Mat1_StartAcquisition.png}
    \caption{The button to start spectrum acquisition in Maestro, highlighted by a red box.}
    \label{fig:mat-start-acquisition}
\end{figure}

\begin{figure}
    \centering
    \includegraphics[width=\figwidth]{figs/Mat2_MCA-Spectrum.png}
    \caption{An example of how Maestro displays an acquired spectrum. Clicking on the spectrum places a line over the nearest channel, revealing the channel number and the number of counts recorded under that channel.}
    \label{fig:mat-sample-spectrum}
\end{figure}

\begin{figure}
    \centering
    \includegraphics[width=0.4\linewidth]{figs/Mat3_peak-id.png}
    \caption{Peak information as displayed by Maestro, found by right-clicking on a peak and selecting ``Peak Info''}
    \label{fig:mat-peak-info}
\end{figure}

In this way, we recorded the centroid of the peak of the spectrum produced using a pulse amplitude voltage approximately 25\% of the maximum, then repeated for 3 more pulse amplitudes, each time recording the acquired spectrum and the centroid of the identified peak.

\subsection{Experiment 7: Statistical Uncertainty of a Series of Independent Measurements}

In this experiment, we measured two series of independent radiation counts, using the experimental setup shown in Figure~\ref{fig:mat-exp7-setup}. \\

\begin{figure}
    \centering
    \includegraphics[width=\figwidth]{figs/Mat4_Exp7-setup.png}
    \caption{The setup for experiment 7. The connection to the Oscilloscope is dotted to signify that while the Oscilloscope may be used to debug signals sent to the Counter, it is not crucial to the experiment and may be removed afterwards.}
    \label{fig:mat-exp7-setup}
\end{figure}


After acquiring the Source from the lab instructors, we removed the cap from the detector and taped the Source to the detector front, with the sticker side of the Source facing the center of the detector.
We then ensure all equipment is connected as in Figure~\ref{fig:mat-exp7-setup}.
With assistance from the TA to choose an appropriate bias setting for the detector, we then set the Power Supply to this setting, slowly incrementing the knob one setting at a time to avoid the risk of damaging the detector. (Specifically, using the clear dial at the top: Rotating the dial either changed a digit's value or selected a new digit, with pressing the dial toggling between these two nodes. While selecting the ``VSET'' option, the former function instead switched between setting voltage bias and ramping the voltage to its new setting) \\

Once the detector was ready, we then set the timer on the Counter to 30 seconds, started the timer and recorded the number of counts after completion. 
We repeated this process for 19 further trials, recording the counts from a total of 20 trials each lasting 30 seconds.
Additionally, in the same fashion we performed a second series, with 200 trials each measuring counts over 5 seconds.
We compared these series of radiation counts to poisson, normal, and binomial distributions fit to these series of sample data.

\subsection{Experiment 8: Accuracy of a Single Measurement}

In this experiment we performed one source and one background measurement to estimate the net activity of the Source.\\

We began by reusing the setup for experiment 7, shown in Figure~\ref{fig:mat-exp7-setup}. 
We set the Counter to record for 5 minutes (300 seconds) with the Source still attached to the detector, and recorded the number of counts measured after this time as the gross count.\\

We then removed the beta source from the detector, returning it to the TA and ensuring there were no sources remaining near the detector.
Setting the counter to record for 10 minutes (600 seconds) with the beta source no longer attached to the detector, we recorded the number of counts measured after this time as the background count.
After completing these measurements, we slowly ramped down the bias on the power supply.\\

Using the gross and background counts, we estimated a net activity and calculated the associated uncertainty, to compare with the calibrated activity of the beta source.\\


\section{Experimental Results}\label{sec:results}

%%%% Experiment 6: Multichannel Analyzer
\FloatBarrier\subsection{Experiment 6: Multichannel Analyzer}

% Plot of the acquired spectrum
\begin{figure}[!h]
    \centering
    \includegraphics[width=\figwidth]{figs/Exp6_4vSpectrum.png}
    \caption{The spectrum acquired from the Multichannel Analyzer with a pulse amplitude of 4V.}
    \label{fig:exp6-4v-spec}
\end{figure}

% "Plots of sample spectra as a function of the pulse amplitude"
\begin{figure}[!h]
    \centering
    \includegraphics[width=\figwidth]{figs/Exp6_V_v_chan.png}
    \caption{The peak centroid of spectra acquired from the Multichannel Analyzer with pulse amplitudes of varying voltage.}
    \label{fig:exp6-v-v-chan}
\end{figure}

%%%% Experiment 6: Repeated Measurements
\FloatBarrier\subsection{Statistical Uncertainty of a Series of Independent Measurements}

Two series of count measurements were performed, of 20 trials lasting 30 seconds each and of 200 trials lasting 5 seconds each.
For both data series, we calculate the sample mean and standard deviation, and use Method of Moments (MoM) to fit experimental results to theoretical Normal, Poisson, and Binomial distributions.\\

For each series, we calculate experimental mean $\bar{x}$ (where $n\in\{20,200\}$ is the number of measurements, and $x_i$ the (count) value measured for trial $i$):
\begin{equation}
    \bar{x} = \frac{1}{n}\sum_{i=1}^{n}x_i
\end{equation}
We then calculate the deviation $d_i$ of each point $x_i$:
\begin{equation}
    d_i = x_i - \bar{x}
\end{equation}
Using these deviations we calculate the sample variance\footnote{As this is the variance of a \emph{sample} and not a known population, the sample mean inherently depends on the specific sample itself, removing a degree of freedom. We thus normalize deviation by $n-1$ rather than by $n$.} $s^2$:
\begin{equation}
   s^2 = \frac{1}{n-1}\sum_{i=1}^{n}d_i 
\end{equation}
And sample standard deviation:
\begin{equation}
    s = \sqrt{s^2}
\end{equation}
For the 20 trials lasting 30 seconds, a mean count of 364.6 was record, with variance 277.727 counts. 
For the 200 trials lasting 5 seconds, a mean of 58.705 counts were recorded, with variance of 49.315 counts.

Using these two `moments', we fit theoretical Normal, Poisson, and Binomial distributions to the data using Method of Moments\cite{kotz_encyclopedia_2006}.
For the Poisson distribution $Pois(\lambda)$, the theoretical mean $\mu$ is equivalent to the frequency parameter $\lambda$, and we fit ($\hat{\lambda}$) using the sample mean as such:
\begin{align}
    \mu &= \lambda = \bar{x}\\
    \hat{\lambda} &= \bar{x}
\end{align}
For the Normal distribution $Norm(\mu, \sigma)$ the moments mean $\mu$ and standard deviation $\sigma$ are themselves parameters, and may be fit to sample values as such:
\begin{align}
    \hat{\mu} &= \bar{x} \\
    \hat{\sigma} &= s \\
\end{align}
For the Binomial distribution $Binom(n, p)$ these moments are functions of parameters $n$ and $p$, and may be inverted to fit to sample moments:
\begin{align}
    \mu &= np = \bar{x} \\
    \sigma &= np(1-p) &= s \\
    &... \\\
    \hat{n} &= \mathrm{round}\left(\frac{\bar{x}^2}{\bar{x} - s^2}\right) \\
    \hat{p} &= \bar{x}/\hat{n}
\end{align}

Histograms of both series' (20 trials lasting 30 seconds each and 200 trials lasting 5 seconds each) results are displayed and compared to analogous histograms of theoretical distributions in Figures~\ref{fig:exp7-30s-hist-v-fits} and \ref{fig:exp7-5s-hist-v-fits}, respectively.

% Histogram of the counts as explained in the text.
% 20 trials, 30 seconds each
\begin{figure}[!h]
    \centering
    \includegraphics[width=\figwidth]{figs/Exp7_20x30_hist_v_fits.png}
    \caption{
        A histogram of the counts observed over 20 trials each lasting 30 seconds, compared to expected counts under theoretical Normal, Poisson, and Binomial distribution distributions.
        Theoretical distributions were fit using Method of Moments.
    }
    \label{fig:exp7-30s-hist-v-fits}
\end{figure}
% 200 trials, 5 seconds each
\begin{figure}[!h]
    \centering
    \includegraphics[width=\figwidth]{figs/Exp7_200x5_hist_v_fits.png}
    \caption{
        A histogram of the counts observed over 200 trials each lasting 5 seconds, compared to expected counts under theoretical Normal, Poisson, and Binomial distribution distributions.
        Theoretical distributions were fit using Method of Moments.
    }
    \label{fig:exp7-5s-hist-v-fits}
\end{figure}

Notably, the distribution of 200 5-second trials in Figure~\ref{fig:exp7-5s-hist-v-fits} much more closely resembles theoretical distributions, forming a stable bell curve in contrast to the bimodal distribution seen in Figure~\ref{fig:exp7-30s-hist-v-fits} of the 20 30-second trials.
Additionally, the Normal and Bimodal distributions almost entirely overlap; this is expected, as at high $n$, Binomial distributions approach that of a Normal distribution with equivalent mean and standard deviation\cite{kotz_encyclopedia_2006}.


% Experiment 8: The Accuracy of a Single Measurement
\FloatBarrier\subsection{Experiment 8: The Accuracy of a Single Measurement}

Detector pulses were counted over a period of 5 minutes while attached to the Source for 5 minutes, and again without the Source for 10 minutes.\\

Over the 5-minute (300 second) Source period, a gross count $n_G$ of 3469 was observed, with an uncertainty of $\sigma_G=\sqrt{n_G}=58.90$.
This yields a gross count rate $f_G=3469\times(300\mathrm{s})^{-1}=11.562\mathrm{s^{-1}}$.
Since the error of a scaled function $a\times A$ is equivalent to its error scaled $a\times \sigma_A$\cite{tsoulfanidis_measurement_2015}, this yields a frequency uncertainty $\sigma_{f,G}=300^{-1}\times58.90=0.196\mathrm{s^{-1}}$.\\

Over the 10-minute (600 second) Source-free period, a background count $n_B$ of 386 was observed, with an uncertainty of $\sigma_B=\sqrt{n_B}=19.647$. 
This yields a background count rate $f_B=386\times(600\mathrm{s})^{-1}=0.643\mathrm{s^{-1}}$, and an error $\sigma_{f,B}=600^{-1}\times19.647=0.00739\mathrm{s^{-1}}$.\\

Assuming infinite timing precision, we may estimate the net activity rate $f_N$ by subtracting $f_B$ from $f_G$:
\begin{align}
    f_N &= f_G - f_B \\
    &= 11.562 - 0.643 \\
    f_N &= 10.920 \mathrm{s^{-1}}
\end{align}
As the variance of a difference $A-B$ is equivalent to the sum of each quantity's variance $\sigma_A^2+\sigma_B^2$\cite{tsoulfanidis_measurement_2015}, we may quantify the uncertainty of this measurement:
\begin{align}
    \sigma_f &= \sqrt{\sigma_{f,G}^2 + \sigma_{f,B}^2} \\
    &= \sqrt{0.196^2 + 0.007^2} \\
    \sigma_f &= 0.1964 \mathrm{s^{-1}}\\
\end{align}

The Source's calibrated activity was 1.0 microCurie\cite{npre_lab_2_manual_2025}, equivalent to $3.7\times10^{10}\mathrm{s^{-1}}\times\mathrm{\frac{1Ci}{10^6\mu Ci}}=37,000\mathrm{s^{-1}}$.
Notably, this is dramatically higher than the net \emph{observed} activity of $10.920\pm0.196 \mathrm{s^{-1}}$.


\FloatBarrier\section{Discussion}\label{sec:disc}
\section{Conclusions}\label{sec:conc}

\appendix
\section{Equipment Details}\label{app:equip}

\newcommand{\na}{$N/A$}
\newcommand{\sref}[1]{$^{\ref{#1}}$}
\FloatBarrier
Further details of equipment are listed in Table~\ref{tab:equip}.
Where a detail could not be identified for a given device, it is replaced with ``\na''.
\begin{table}[!h]
    \centering
    \small
    \caption{Details of the equipment used in this laboratory.}
    \label{tab:equip}
    \scriptsize
    \begin{tabular}{c|cccc}
        Item & Manufacturer & Model & ID Type & ID Number\\\hline
        Module rack & Canberra (now Mirion)\sref{addr:mirion}& Model 2000 & Inventory \# & NE759377 \\
        Geiger-Muller Detector & \na & \na & \na & \na \\
        Multichannel Analyzer & Ortec\sref{addr:ortec} & EasyMCA & Inventory \# & P10F79090 \\
        Radiation Source & Spectrum Technologies\sref{addr:spect} & 2017 1.0\micCi $^{137}$Cs Beta Source & Number & 2037 \\
        High Voltage Power Supply & Caen\sref{addr:caen} & N1470AL & Inventory \# & P10G96054 \\
        Pulser & Ortec\sref{addr:ortec} & Model 480 & \na & \na \\
        Oscilloscope & Keysight\sref{addr:keys} & Infiniivision DSO-X 2002A & Inventory \# & P10R03513 \\
        Counter & Ortec\sref{addr:ortec} & Model 871 & Serial \# & 1024 \\
        Preamplifier & Ortec\sref{addr:ortec} & Model 142PC & Serial \# & 2780r17 \\
        Amplifier & Ortec\sref{addr:ortec} & Model 590A & Serial \# & 16076205 \\
        Computer & Dell\sref{addr:dell} & Precision 360 & Eng. IT ID \# & EWS 20426 \\

    \end{tabular}
    \begin{enumerate}[label=\alph*]
        \item\label{addr:mirion}
        Mirion Technologies, Inc.:
        1218 Menlo Drive,
        Atlanta, GA;
        Phone: 770.432.2744;
        \url{https://ir.mirion.com/}
        \item\label{addr:ortec}
        Advanced Measurement Technology:
        801 South Illinois Avenue,
        Oak Ridge, Tennessee 37830,
        United States;
        Phone: +1.865.482.4411;
        Fax: +1.865.483.0396;
        \url{https://www.ortec-online.com};
        Email: ortec.info@ametek.com
        \item\label{addr:spect}
        Spectrum Techniques:
        106 Union Valley Rd, Oak Ridge, TN 37830;
        Phone: 865.482.9937;
        \url{https://www.spectrumtechniques.com/}
        \item\label{addr:caen}
        Caen Sp.A.:
        Via della Vetraia, 11, 55049 Viareggio LU - Italy;
        Phone: +39 0584 388398;
        \url{https://www.caen.it/}
        \item\label{addr:keys}
        Keysight Technologies: 
        1900 Garden of the Gods Road, Colorado Springs, CO 80907-3423 United States;
        Phone: 1 800 829-4444;
        \url{https://www.keysight.com}
        \item\label{addr:dell}
        Dell Technologies:
        Dell Technologies
        1 Dell Way
        Round Rock, TX 78664.;
        Phone: 1-877-275-3355;
        \url{https://www.dell.com/en-us/lp/contact-us}
    \end{enumerate}
\end{table}
\FloatBarrier

% \bibliographystyle{sty/ieeetran}
\printbibliography
\end{document}
